% Created 2023-03-30 Thu 17:04
% Intended LaTeX compiler: pdflatex
\documentclass[a4paper, 12pt]{article}
\usepackage[utf8]{inputenc}
\usepackage[T1]{fontenc}
\usepackage{graphicx}
\usepackage{longtable}
\usepackage{wrapfig}
\usepackage{rotating}
\usepackage[normalem]{ulem}
\usepackage{amsmath}
\usepackage{amssymb}
\usepackage{capt-of}
\usepackage{hyperref}
\usepackage{authblk}
\usepackage{breakcites}
\usepackage{apacite}
\usepackage[top=3cm, bottom=3cm, left=3cm, right=3cm]{geometry} % define (reduced) margin size
\usepackage[parfill]{parskip} % insert whitespace between new paragraphs
\setlength\parindent{0pt}
\tolerance=9999
\emergencystretch=10pt
\hyphenpenalty=10000
\exhyphenpenalty=100
\author[1]{Patrick O'Toole}
\author[1,2]{Matthew L. Thomas}
\author[1]{Oliver Stevens}
\author[1,3]{Kevin Lam}
\author[4]{Katherine Kripke}
\author[1]{Rachel Esra}
\author[5]{Ian Wanyeki}
\author[5]{Lycias Zembe}
\author[1]{Jeffrey W. Eaton}
\affil[1]{\emph{Imperial College London, London, United Kingdom}} \\
\affil[2]{\emph{Joint Centre for Excellence in Environmental Intelligence, University of Exeter and Met Office}} \\
\affil[3]{\emph{Department of Statistics, University of British Columbia}} \\
\affil[4]{\emph{Avenir Health, Takoma Park, MD, USA}} \\
\affil[5]{\emph{Joint United Nations Programme on HIV/AIDS (UNAIDS)}} \\
\date{17th October 2022}
\title{District-level male medical and traditional circumcision coverage and unmet need in sub-Saharan Africa}
\begin{document}

\maketitle

\clearpage

\begin{abstract}
  \noindent \textbf{Background} \\
  \noindent In 2016, UNAIDS developed a Fast-Track strategy that targeted 90\% coverage
  male circumcision (MC) among men aged 10-29 years by 2021 in priority countries in sub-Saharan 
  Africa (SSA) to reduce HIV incidence. There is substantial variation across subnational 
  regions within countries in both traditional male circumcision (TMC) practices and progress
  towards implementation of voluntary medical male circumcision (VMMC). Tracking progress and
  remaining gaps towards VMMC HIV prevention targets requires detailed district-level circumcision
  coverage data.

  \noindent \textbf{Methods} \\
  \noindent We analysed self-reported data on male circumcision from x nationally representative household
  surveys conducted in x SSA countries between 2006-2020. A spatio-temporal Bayesian
  competing-risks time-to-event model was used to estimate rates of traditional and medical
  circumcision by age, location, and time. Circumcision coverage in 2020 was projected assuming
  continuation of estimated age-specific rates, with probabilistic uncertainty.

  \noindent \textbf{Results} \\
  \noindent Across x countries, from 2010 to 2020 an estimated x million men (x\% CI x-x million)
  were newly circumcised, of whom x million (x - x million) were medically circumcised, and
  x million (x - x million) traditionally circumcised. In 2020, MC coverage among men 10-29
  years ranged from x\% (x\% - x\%) in Zimbabwe (?) to x\% (x\%-x\%) in Togo. MMC coverage
  ranged from x\% (x\%-x\%) in Malawi to x\% (x\%-x\%) in Tanzania, and TMC coverage
  from x\% (x\%-x\%) in Eswatini to x\% (x\%-x\%) in Ethiopia. The largest increase in MMC
  coverage was in Lesotho from x\% to x\%. Within countries, the median difference in MC
  coverage between the districts with lowest and highest coverage was x\%, with the smallest
  variation in Eswatini (x\% to x\%) and largest in Zambia (x\% to x\%). x million men aged
  10-29 need to be circumcised to reach 90\% coverage in all countries.

  \noindent \textbf{Conclusions} \\
  \noindent VMMC programmes have made substantial, but uneven, progress towards male circumcision targets. Granular district and age-stratified data provide information for focusing further programme implementation.

\end{abstract}

\newpage

\section{Background}
\label{sec:org3a33a7d}

HIV remains the single largest cause of years of life lost among adolescent boys and men of reproductive age in eastern and southern Africa.

Voluntary Male medical circumcision (VMMC) reduces the rate of male-to-female HIV incidence by 60\%.
Efficient, cost-effective, one-time procedure for preventing HIV transmission.

Matt's model \ldots{} (Thomas, Matthew L. and Zuma, Khangelani and Loykissoonlal, Dayanund and Dube, Bridget and Vranken, Peter and Porter, Sarah E. and Kripke, Katharine and Seatlhodi, Thapelo and Meyer-Rath, Gesine and Johnson, Leigh F. and Eaton, Jeffrey W., 2021)

\newpage
\section{Data}
\label{sec:org0422724}

120 household surveys conducted in 33 SSA countries 2002-2019
Self-reported circumcision:
o Status (MC vs uncircumcised), 
o Type (MMC vs TMC), 
o Year, and
o Age 
recorded
Sub-national populations from WorldPop (reference)

Major survey series (DHS, AIS, PHIA, MICS, HSRC in ZAF)
Individual-level data: self-reported circumcision status  by male respondents
Respondents located to districts using cluster geocoordinates
Located to admin 1 (province) where coordinates not available (MICS)
VMMC programme data not used

Circumcisions performed by a medical professional and/or in a medical setting are categorised as MMC
Otherwise, circumcisions are TMC
Where no data is present on location or provider, circumcision type  Missing

Individual-level household survey data provide direct estimates of circumcision rates over time and by type for years preceding survey

 Direct estimates of TMC practices, age at circumcision, VMMC impact



\newpage

\section{Methods}
\label{sec:org69a085f}

\subsection{Model}
\label{sec:org2077286}

Bayesian spatio-temporal, competing-risks, time-to-event model
Stratified by age, location and time
Rates of TMC and medical male circumcision (MMC) estimated
Coverage in 2020 projected assuming continuation of estimated age-specific rates with probabilistic uncertainty
Important assumption: Probability of traditional male initiation ceremonies (TMICs) constant over time (needed? Might lead to a lot of questions!) 

Model stratified by:
Age
District
circumcision type (traditional / medical)

TMC \& MMC rates estimated (by age, district, and time)
Spatial smoothing allows for district level estimates

Circumcision coverage since most recent HH survey: projected assuming continuation of estimated age-specific rates, with probabilistic uncertainty

Important assumption: TMC rate assumed constant over time


\newpage

Samples drawn from model fit, weighted by population and aggregated, to produce estimates for less granular regions (e.g. District to Province aggregation).
Computed posterior summary statistics for coverage, circumcision incidence and probability of being circumcised. 
Done for all strata, for both discrete ages and “binned” age groups.



\section{Results}
\label{sec:org634cf32}

\#+latex \newpage

\section{Discussion}
\label{sec:orgce42292}

\textbf{Challenges}
Inconsistent MC self-reporting by same cohort in successive surveys
E.g. in 2017 survey, men 30-34 report higher \% circumcised in 2012 than ‘same’ men age 25-29 in 2012 survey
Affects circumcision level, and distribution by type

‘Replacement’ of traditional circumcision by medical circumcision
Evidence of this in surveys from several countries; work in progress
Also not fully accounted for in DMPPT2 baseline coverage inputs

Surveys imply different level of scale-up than programme data
Several countries: surveys suggest fewer VMMCs conducted than programme data

\#+latex \newpage

\section{References}
\label{sec:org38b7f8b}

\#+PRINT\textsubscript{BIBLIOGRAPHY}
\end{document}